\newpage

% \newtitleanum{Anexos}

\section{Anexos}

\insertimage[\label{im:diagramalineal}]{exp2_pendulo_LQR.pdf}{width=0.7\linewidth}{Diagrama de bloques \textit{Simulink} para modelo lineal y modelo no lineal.}


\insertimage[\label{im:simulink_obs1}]{exp2_pendulo_LQR_observer.pdf}{page=1, width=0.7\linewidth}{Implementación \textit{Simulink} sistema lineal y no lineal con observador.}

\insertimage[\label{im:simulink_obs2}]{exp2_pendulo_LQR_observer.pdf}{page=2, width=0.7\linewidth}{Subbloque planta No lineal \textit{Simulink} sistema lineal y no lineal con observador.}

\insertimage[\label{im:simulink_obs3}]{exp2_pendulo_LQR_observer.pdf}{page=3, width=0.7\linewidth}{Subbloque observador lineal \textit{Simulink} sistema lineal y no lineal con observador.}

\insertimage[\label{im:simulink_obs4}]{exp2_pendulo_LQR_observer.pdf}{page=4, width=0.7\linewidth}{Subbloque observador No lineal \textit{Simulink} sistema lineal y no lineal con observador.}

% \begin{figure}[H]
%     \centering
%     \includegraphics[page=1, width=0.99\linewidth]{images/exp2_pendulo_LQR_observer.pdf}
%     \label{im:simulink_obs1}
%     \caption{Implementación \textit{Simulink} sistema lineal y no lineal con observador.}
% \end{figure}

% \begin{figure}[H]
%     \centering
%     \includegraphics[page=2, width=0.99\linewidth]{images/exp2_pendulo_LQR_observer.pdf}
%     \label{im:simulink_obs2}
%     \caption{Subbloque planta No lineal \textit{Simulink} sistema lineal y no lineal con observador.}
% \end{figure}

% \begin{figure}[H]
%     \centering
%     \includegraphics[page=3, width=0.99\linewidth]{images/exp2_pendulo_LQR_observer.pdf}
%     \label{im:simulink_obs3}
%     \caption{Subbloque observador lineal \textit{Simulink} sistema lineal y no lineal con observador.}
% \end{figure}

% \begin{figure}[H]
%     \centering
%     \includegraphics[page=4, width=0.99\linewidth]{images/exp2_pendulo_LQR_observer.pdf}
%     \label{im:simulink_obs4}
%     \caption{Subbloque observador No lineal \textit{Simulink} sistema lineal y no lineal con observador.}
% \end{figure}




% \insertimage[\label{im:simulink_obs}]{exp2_pendulo_LQR.pdf}{width=0.7\linewidth}{Implementación en \textit{Simulink} de observadores de estado.}

% \lstset{style=Python}
% \lstinputlisting[language=python, basicstyle=\fontsize{9}{9}\ttfamily, caption={Implementación 
% preprocesamiento.}]{cod/data_cleaning_paper.py}


% \lstset{style=C}
% \lstinputlisting[language=C, basicstyle=\fontsize{9}{9}\ttfamily, caption={Código completo de la implementación.}]{cod/tarea6.c}

% \lstinputlisting[language=C, basicstyle=\fontsize{9}{9}\ttfamily, caption={\textit{header} con el prototipo.}]{cod/func.h}

% \lstset{style=python}
% \lstinputlisting[language=python,basicstyle=\tiny, caption={Código para hacer las comparaciones entre algoritmos en secuencias aleatorias, ascendentes y descendentes.}]{cod/order_measure.java}


%\begin{figure}[H] \centering
%\begin{subfigure}{.75\textwidth}
%  \centering
%  \includegraphics[width=.75\linewidth]{images/varpar/acceptedRansac_c5.jpg}
%  \caption{$err_{min}=60$, $cons_{min}=5$, $n_{trials}=10000$.}
%  \label{fig:an111}
%\end{subfigure}\\%
%\begin{subfigure}{.75\textwidth}
%  \centering
%  \includegraphics[width=.75\linewidth]{images/varpar/acceptedRansac_c50.jpg}
%  \caption{$err_{min}=60$, $cons_{min}=50$, $n_{trials}=10000$.}
%  \label{fig:an112}
%\end{subfigure}
%\caption{RANSAC: Variación de $umbralcons$.}
%\label{fig:an11_}
%\end{figure}

